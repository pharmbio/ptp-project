
\documentclass[10pt,article]{memoir}

\usepackage[utf8]{inputenc}
\usepackage{hyperref}
\usepackage{microtype}
\usepackage{letltxmacro}
\usepackage[lining]{ebgaramond}
\usepackage[cmintegrals,cmbraces]{newtxmath}
\usepackage{ebgaramond-maths}
\usepackage{booktabs}
\usepackage{mathtools}
\usepackage{amssymb}
\usepackage{tikz}
\usepackage{enumitem}
\usepackage{multirow}
\usepackage{rotating}
\renewcommand{\thefootnote}{\color{red}\arabic{footnote}}
\usepackage{placeins}
\usepackage[textsize=small, linecolor=magenta, bordercolor=magenta,
            backgroundcolor=magenta, textwidth=3cm]{todonotes}

\urlstyle{sf}
\makeatletter
    \let\UrlSpecialsOld\UrlSpecials
    \def\UrlSpecials{\UrlSpecialsOld\do\/{\Url@slash}\do\_{\Url@underscore}}%
    \def\Url@slash{\@ifnextchar/{\kern-.11em\mathchar47\kern-.2em}%
        {\kern-.0em\mathchar47\kern-.08em\penalty\UrlBigBreakPenalty}}
        \def\Url@underscore{\nfss@text{\leavevmode \kern.06em\vbox{\hrule\@width.3em}}}
\makeatother
\captionnamefont{\small\bfseries}
\captiontitlefont{\small}

\makeatletter
    \renewcommand{\@todonotes@drawMarginNoteWithLine}{%
    \begin{tikzpicture}[remember picture, overlay, baseline=-0.75ex]%
        \node [coordinate] (inText) {};%
    \end{tikzpicture}%
    \marginpar[{% Draw note in left margin
        \@todonotes@drawMarginNote{r}%
        \@todonotes@drawLineToLeftMargin%
    }]{% Draw note in right margin
        \@todonotes@drawMarginNote{l}%
        \@todonotes@drawLineToRightMargin%
    }%
    }
    \renewcommand{\@todonotes@drawMarginNote}[1]{
        \makebox[\marginparwidth][#1]{\begin{tikzpicture}[remember picture,baseline=(X.base)]%
            \node(X){\vphantom{X}};%
            \draw node[notestyle,font=\@todonotes@sizecommand,anchor=north] (inNote) at (X.north)%
                {\@todonotes@text};%
            \if@todonotes@authorgiven%
                \draw node[notestyle,font=\@todonotes@sizecommand,anchor=north] (inNote) at (X.north)%
                    {\@todonotes@sizecommand\@todonotes@author};%
                \node(Y)[below=of X]{};%
                \draw node[notestyle,font=\@todonotes@sizecommand,anchor=north] (inNote) at (X.south)%
                    {\@todonotes@text};%
            \else%
                \draw node[notestyle,font=\@todonotes@sizecommand,anchor=north] (inNote) at (X.north)%
                    {\@todonotes@text};%
            \fi%
        \end{tikzpicture}%
    }}
\makeatother
\LetLtxMacro{\oldtodo}{\todo}
\renewcommand{\todo}[1]{{\color{magenta}\oldtodo[fancyline]{\color{white}\textsf{#1}}}}
\newcommand{\inlinetodo}[1]{{\color{magenta}\oldtodo[inline]{\color{white}\textsf{#1}}}}

\setsecheadstyle{\LARGE}
\setsubsecheadstyle{\large}
\setsubsubsecheadstyle{\itshape}
\setparaheadstyle{\normalsize\scshape\liningnums}
\counterwithout{figure}{chapter}
\counterwithout{table}{chapter}
\captionnamefont{\textsf\small}
\captiontitlefont{\textsf\small}
\let\thempfootnote\thefootnote

\newsubfloat{figure}
\mergepagefloatstyle{floatcomp}{plain}{empty}
\pagestyle{floatcomp}

\newenvironment{wideMinipage}
{ \vskip 1\baselineskip
  \noindent   
  \checkoddpage% 
  \ifoddpage%
     \hspace*{-3em}%
  \else%
     \hspace*{-3em}%
  \fi%
  \begin{minipage}{1\textwidth + 6em}
}
{ 
    \end{minipage}
    \vskip 1\baselineskip
}

\begin{document}

\begin{center}
{\LARGE\noindent Predicting off-target binding profiles with confidence using Conformal
Prediction}
\end{center}

\begin{center}
\noindent
\begin{tabular}{ccc}
Samuel Lampa\,$^{1,*}$  & Jonathan Alvarsson\,$^{1}$ &  Staffan Arvidsson Mc Shane\,$^{1}$ \\
 Arvid Berg\,$^{1}$ &Ernst Ahlberg\,$^{2}$\, & Ola Spjuth\,$^{1}$ \\
\end{tabular}
\end{center}

\noindent
\begin{minipage}{1.1\textwidth}
\noindent\footnotesize $^1$Pharmaceutical Bioinformatics group, Department of Pharmaceutical Biosciences, Uppsala University, Uppsala, Sweden \\ 
\noindent\footnotesize $^2$Predictive Compound ADME \& Safety, Drug Safety \& Metabolism, AstraZeneca IMED Biotech Unit, Mölndal, Sweden \\ 
\end{minipage}


\def\keyFont{\fontsize{8}{11}\helveticabold }
\def\firstAuthorLast{Lampa {et~al.}} %use et al only if is more than 1 author
\def\Authors{Samuel Lampa\,$^{1,*}$, Jonathan Alvarsson\,$^{1}$, Staffan Arvidsson Mc Shane\,$^{1}$, Arvid Berg\,$^{1}$, Ernst Ahlberg\,$^{2}$  and Ola Spjuth\,$^{1}$}
\def\Address{$^{1}$Pharmaceutical Bioinformatics group, Department of Pharmaceutical Biosciences, Uppsala University, Uppsala, Sweden\\
$^{2}$Predictive Compound ADME \& Safety, Drug Safety \& Metabolism, AstraZeneca IMED Biotech Unit, M"olndal, Sweden}
\def\corrAuthor{Corresponding Author}

\def\corrEmail{samuel.lampa@farmbio.uu.se}

\begin{abstract}
\noindent
Panels of ligand-based models are widely used in drug discovery to obtain an
early indication of potential off-target predictions that could be linked
to drug efficacy and safety. Another application is to combine models into
a profile allowing to compare and search for compounds with similar
profile. Most contemporary methods and implementations however lack valid
measures of confidence in predictions, yielding only point predictions. We
here describe the use of conformal prediction for predicting off-target
interactions based on data from the ExCAPE database about 31 binding
targets, selected for their value in broad early hazard assessment. We used
support vector machines, with chemical described by the signature molecular
descriptor, to train predictive models.  Prediction results from the
resulting models are presented as prediction intervals, as customary in
conformal prediction. The full pre-processing and model training process is
openly available as a scientific workflow on github, rendering it fully
reproducible. The resulting models are published online and are available
via a graphical web interface and an OpenAPI interface for programmatic
access.

%\tiny
 %\keyFont{ \section{Keywords:} drug discovery, predictive modelling, conformal prediction, machine learning, off-target, adverse effects} %All article types: you may provide up to 8 keywords; at least 5 are mandatory.
\end{abstract}

\section*{Introduction\todo{Expand the intro, prior art?}}

%Introduce off target binding as a problem, and the value of predictions

Off-target pharmacology and polypharmacology has big implications on drug
efficacy and safety, and prediction of target binding profiles for ligands is
an important task that can aid early drug discovery~\cite{Bowes2012}. Drugs
often does not interact with just one target but with multiple
targets~\cite{hopkins2008network}. However, available methods for ligand-based
target profiling often do not offer valid measures of confidence in
predictions. We present an approach for ligand-based target profiling using a
confidence framework, delivering target profiles with confidence score for the
predictions of whether a query compound interact with each target. The
confidence scores are calculated using the Conformal Prediction
methodology~\cite{Vovk2005}, with support vector machines for modeling and
chemical structures described by the molecular signature
descriptor~\cite{Faulon2003}. The molecular signatures have previously been
successfully used for ligand-based target prediction~\cite{alvarsson2014ligand}

The goal of this study was to create an automated and reproducable approach for
generating a predicted target profile by QSAR binding models with the models
making up the profile published online as micro services and the profile
accessible from a web page. Although the models give a confidence measure we
also set out to evaluate them on test sets to see how well they perform on some
representable data. We exemplified the process by creating a profile on the
targets for broad early hazard assessment presented by Bowes \textit{et
al.}~\cite{Bowes2012}.

%%%%%%%%%%%%%%%%%%
%Methods
%%%%%%%%%%%%%%%%%%
\section*{Methods}

\subsection{Training data}

A scientific workflow was constructed to automate the entire data preprocessing.
The first step comprised extracting data on binding association between ligands and targets from the database ExcapeDB~\cite{Sun2017}, more specifically the columns Gene name, Original entry ID (PubChem CID or CHEMBL ID), SMILES and Activity flag. This was performed early in the workflow to make subsequent data transformation steps more performant, given the relative large size of the
uncompressed ExcapeDB data file (18 GB).
%
From the extracted dataset, all rows for which there existed rows with a conflicting
activity value, were completely removed.
% (See \footnote{\url{https://github.com/pharmbio/ptp-project/blob/c529cf/exp/20180426-wo-drugbank/wo\_drugbank\_wf.go\#L239-L246}}).

A subset of the panel of 44 binding targets as suggested in \cite{Bowes2012}
was selected for inclusion in the study. The selection was based on that targets
should have at least 100 associated and at least 100 non-associated compounds. 
In addition some targets were excluded for which data was not found in ExcapeDB, this is described in detail below.
%
Some of the gene ids used in \cite{Bowes2012} were not found in their exact
form in the ExcapeDB dataset. To resolve this, PubMed was consulted to find
synonymous gene IDs with the following replacements done:
%
\textit{KCNE1} was replaced with \textit{MINK1} which is present in ExcapeDB.
\textit{CHRNA1} (coding for the $\alpha1$ subunit of the Acetylcholine
receptor) is excluded, as it is not present in the dataset (\textit{CHRNA4},
coding for the $\alpha4$ subunit of the Acetylcholine receptor, is present in
the dataset). We note that both \textit{MINK1} and \textit{CHRNA4} were removed in the filtering
step mentioned above, since the dataset did not contain more than 100 active
and 100 non-active compounds for \textit{MINK1} nor \textit{CHRNA}. 
However, since one aim of the study is to present and publish an automated and
reproducible data processing, these targets could potentially be included in
subsequent runs on later versions of the database with additional data
available.

The resulting set (named Dataset1) consisted of 31 targets (the upper half of Table~\ref{tbl:targets}).
For 21 of these targets, the dataset contained less than 10\,000 non-active
compounds, leading to imbalanced datasets (marked with a \checkmark\ in the
`Filled-up' column of Table~\ref{tbl:targets}).
%
For these 21 targets, we filled up their respective datasets with randomly
selected examples from the raw dataset which were not reported to be active for
this target, thus being 'assumed non-active'. The number of new examples was chosen such that the total number of non-actives and
assumed non-actives would be twice the number of actives, for each target
respectively. This dataset was named Dataset2.
% (See \href{https://github.com/pharmbio/ptp-project/blob/c529cf307593c40e7f822a92b224036894c95de1/exp/20180426-wo-drugbank/wo_drugbank_wf.go#L308-L328}{here} for workflow code).
%
All the targets, with details about their respective number of active
and non-active compounds, and whether they are included or not, are summarized
in table~\ref{tbl:targets}.

\begin{table}[p]
\begin{wideMinipage}
\small
\centering
\vspace*{-45pt} % This is cheating, top margin should be holy, but this table is HUGE!
\caption{The panel of targets used in this study, identified by Gene ID. Actives and non-actives refer to the number of ligand interactions marked as active and non-active in ExcapeDB. Included indicates if the target was included in the study or excluded because it did not pass the filtering criteria.}
\label{tbl:targets}
\begin{tabular}{crrrrcl}
\toprule
&             &         & Non-actives       & Non-actives      &              &       \\
&             &         & (before fillup \& & (after fillup \& &              &       \\
&     Gene ID & Actives & deduplikation)    & deduplikation)   & Filled-up  & Remarks \\
\midrule
\multirow{31}{*}{\begin{turn}{90}\textsc{Included}\end{turn}} 
&    ACHE    &       3\,160  &       1\,152      &   5\,824   & \checkmark      &       \\
&    ADORA2A &       5\,275  &       593         &   10\,092  & \checkmark      &       \\
&    ADRB1   &       1\,306  &       149         &   2\,544   & \checkmark      &       \\
&    ADRB2   &       1\,955  &       342\,282    &   341\,925 &       &       \\
&    AR      &       2\,593  &       4\,725      &   4\,866   & \checkmark      &       \\
&    AVPR1A  &       1\,055  &       321\,406    &   321\,098 &       &       \\
&    CCKAR   &       1\,249  &       132         &   2\,458   & \checkmark      &       \\
&    CHRM1   &       2\,776  &       417\,549    &   358\,330 &       &       \\
&    CHRM2   &       1\,817  &       152         &   3\,440   & \checkmark      &       \\
&    CHRM3   &       1\,676  &       144         &   3\,234   & \checkmark      &       \\
&    CNR1    &       5\,336  &       400         &   10\,220  & \checkmark      &       \\
&    CNR2    &       4\,583  &       402         &   8\,676   & \checkmark      &       \\
&    DRD1    &       1\,732  &       356\,201    &   355\,909 &       &       \\
&    DRD2    &       8\,323  &       343\,206    &   342\,958 &       &       \\
&    EDNRA   &       2\,129  &       124         &   4\,050   & \checkmark      &       \\
&    HTR1A   &       6\,555  &       64\,578     &   64\,468  &       &       \\
&    HTR2A   &       4\,160  &       359\,962    &   359\,663 &       &       \\
&    KCNH2   &       5\,330  &       350\,773    &   350\,452 &       &       \\
&    LCK     &       2\,662  &       283         &   5\,246   & \checkmark      &       \\
&    MAOA    &       1\,260  &       1\,083      &   2\,452   & \checkmark      &       \\
&    NR3C1   &       2\,525  &       4\,382      &   4\,804   & \checkmark      &       \\
&    OPRD1   &       5\,350  &       826         &   9\,580   & \checkmark      &       \\
&    OPRK1   &       3\,672  &       303\,335    &   303\,111 &       &       \\
&    OPRM1   &       5\,837  &       2\,872      &   11\,252  & \checkmark      &       \\
&    PDE3A   &       197     &       110         &   392      & \checkmark      &       \\
&    PTGS1   &       849     &       729         &   1\,634   & \checkmark      &       \\
&    PTGS2   &       2\,862  &       827         &   5\,162   & \checkmark      &       \\
&    SCN5A   &       316     &       119         &   624      & \checkmark      &       \\
&    SLC6A2  &       3\,879  &       218         &   7\,498   & \checkmark      &       \\
&    SLC6A3  &       5\,017  &       106\,819    &   106\,594 &       &       \\
&    SLC6A4  &       7\,228  &       382         &   13\,660  & \checkmark      &       \\
\midrule
\multirow{15}{*}{\begin{turn}{90}\textsc{Not Included}\end{turn}} 
&    ADRA1A  &       1\,782  &       24          &            &       &       \\
&    ADRA2A  &       839     &       39          &            &       &       \\
&    CACNA1C &       166     &       20          &            &       &       \\
&    CHRNA1  &       -       &       -           &            &       & Not in ExcapeDB \\
&    CHRNA4  &       256     &       17          &            &       &       \\
&    GABRA1  &       112     &       5           &            &       &       \\
&    GRIN1   &       555     &       92          &            &       &       \\
&    HRH1    &       1\,218  &       65          &            &       &       \\
&    HRH2    &       394     &       56          &            &       &       \\
&    HTR1B   &       1\,262  &       86          &            &       &       \\
&    HTR2B   &       1\,159  &       66          &            &       &       \\
&    HTR3A   &       584     &       65          &            &       &       \\
&    KCNQ1   &       37      &       303\,466    &            &       &       \\
&    MINK1   &       929     &       8           &            &       & Synonym to KCNE1 \\
&    PDE4D   &       484     &       98          &            &       &       \\

\bottomrule
\end{tabular}
\end{wideMinipage}
\inlinetodo{Extend caption}
\end{table}

\subsection{Machine learning, molecular descriptors and conformal prediction\todo{add refs}}
QSAR models are built by describing molecules by molecular descriptors and then
train machine learning methods to, \textit{e.g.} predict target binding for a
chemical. CPSign is a tool for making QSAR models with the molecular
signatures descriptor, support vector machines and conformal prediction. The
molecular signatures descriptor~\cite{faulon2003signature}, is based on the
neighboring of atoms in a molecule and has been shown to work well for QSAR
studies~\cite{lapins2018confidence}. Support vector machines is a machine learning
algorithm which is commonly used in QSAR studies~\ref{norinder2003support,
zhou2011qsar} together with molecular signatures and similar molecular
descriptors, \textit{e.g.}, the extended connecticity fingerprints~\cite{rogers2010extended}. 


\subsection{Hyperparameter tuning}
For each of the 31 targets, a parameter sweep was run to find the optimal value of the
cost parameter of liblinear, optimizing modeling efficiency using 10-fold cross validation. The training
approach used an Aggregated Conformal Predictor (ACP) with 10 aggregated models.
The parameter sweep evaluated three values for the cost parameter for each target; 1, 10 and 100. The
efficiency measure used for the evaluation was the observed fuzziness (OF)
score described in~\cite{Vovk2016} as:

\begin{equation}
OF =\frac{ 1}{m} \sum\limits_{i=1}^{m} \sum\limits_{y_i \neq y }  p_i^{\kern1pt y},		
\end{equation}

where $p_i^{\kern1pt y}$ is the p-value of the $i^{th}$ test case for class $y$, and $m$ is the number of test examples, or in our case with only two classes:

\begin{equation}
OF =  \frac
        {\quad\sum\limits_{\mathclap{\substack{\scriptscriptstyle i,\ y_i=A}}}p_i^N \, + \;\sum\limits_{\mathclap{\substack{\scriptscriptstyle i,\ y_i=N}}}p_i^A}
        {m_A + m_N} 
\end{equation}

where $p_i^N$ is the $i^{th}$ p-value for class $N$, $p_i^A$ is the $i^{th}$
p-value for class $A$ and $m_A$ and $m_N$ is the number of test examples in
class $A$ and $N$ respectively. $OF$ is basicly an average of the p-values for
the wrong class, \textit{i.e.}, lower fuzziness means better prediction.

To study the effect of imbalanced datasets on efficiency, we also implemented a
modified version of OF, due to the fact that OF is influenced more
by values in the larger class in case of imbalanced datasets referred to as
``class-averaged
observed fuzziness'' (CAOF) as:
\begin{equation}
CAOF = \frac
        {\quad\sum\limits_{\mathclap{\substack{\scriptscriptstyle i,\ y_i=A}}}p_i^N}
        {m_A}
       + \frac 
        {\quad\sum\limits_{\mathclap{\substack{\scriptscriptstyle i,\ y_i=N}}}p_i^A}
        {m_N} 
\end{equation}
with the same variable conventions as above. Where $OF$ is only an average for
the p-values in the test set, $CAOF$ takes the contribution from each class
separately, meaning that for very unbalanced cases $OF$ is mainly just affected
by the larger class and $CAOF$ is equally contributed to from both classes.

%We first calculated the OF per class and divided it by the number of examples
%(compounds) in that class, and finally took the mean value of the resulting
%normalized OF values for each class.

%For evaluation of the OF score mentioned above, we ran cross-validation
%through CPSign's built-in crossvalidate function, with 10 folds. 
%(specified to cpsign's crossvalidate and train commands using the
%\texttt{--nr-models} flag).

CAOF was not used for cost selection, but is provided for information in the results from the workflow.


\subsection{Calibration plots}
From the crossvalidation, calibration plots\todo{Write a bit about what a
calibration plot is somewhere?} were generated by predicting values for
confidence values at 0.05 up to 0.95, with a step size of 0.05. These plots are
available in the supplementary material. Additionally, three representative
calibration plots, for the smallest, median-sized, and largest target datasets,
in figure \ref{fig:calibration_plots}.

\begin{figure}[h!]
\includegraphics[width=0.3\textwidth]{figures/calibration_plots/pde3a_calib.pdf}
\includegraphics[width=0.3\textwidth]{figures/calibration_plots/slc6a2_calib.pdf}
\includegraphics[width=0.3\textwidth]{figures/calibration_plots/htr2a_calib.pdf}
    \inlinetodo{Make the axes gradation 0, 0.5, 1.0}
    \caption{Three representative calibration plots, for models PDE3A, SLC6A2
    and HTR2A, based on the smallest, the median, and the largest data sets in
    terms of total number of compounds. The plots show accuracy against
    confidence, for confidence values 0.05 to 0.95 with a step size of 0.05.}
    \label{fig:calibration_plots}
\end{figure}

\subsection{Modeling workflow}
Before the training, the CPSign precompute command was run, in order to
generate a sparse representation of each target's dataset.
ACPs consisting of 10 models were then trained for each target using the CPSign software using the 'train' command.
The cost value used was the one obtained from the hyperparameter tuning. 
% (See \href{https://github.com/pharmbio/ptp-project/blob/c529cf307593c40e7f822a92b224036894c95de1/exp/20180426-wo-drugbank/wo_drugbank_wf.go#L69-L101}{here}).
The computational workflows for orchestrating the extraction of data, model building, 
and the collection of results for summarization and plotting were
implemented in the Go programming language using the SciPipe workflow library 
that is available as open source software at
\href{http://scipipe.org}{scipipe.org} or \href{https://github.com/scipipe/scipipe}{github.com/scipipe/scipipe}.
The cost values for each target is stored in the workflow code, available on
github (https://github.com/pharmbio/ptp-project).  An example of the modeling
workflow is shown in figure \ref{fig:workflow_graph_clean}, which shows the workflow
used for comparison conformal metric when filling up with assumed non-actives versus
not filling up. More detailed workflow graphs are shown in the supplemental material.

\begin{figure}[h!]
\includegraphics[width=\textwidth]{figures/workflow_graph_clean.pdf}
    \caption{Directed graph of processes in the the modeling workflow used to
    generate the models for the comparison of filled up datasets versus
    non-filled up are similar. Each node represents a workflow process (which
    in turn can generate multiple tasks, representing concrete shell command
    incovations). This graph is also slightly simplified by showing multiple
    processes for multiple replicates, as a unified node (indicated by
    \texttt{r\{1,2,3\}} in the node labels.  port names are excluded in this
    graph for readability). Also, here the workflow is only showed for one gene
    (PDE3A), since including all 21 genes (included in the fillup comparison),
    it would render unreadable.}
    \label{fig:workflow_graph_clean}
\end{figure}

\subsection{Model validation}

In order to validate the predictive abilities of the trained models, data about
1\,000 compounds were withheld from the ExcapeDB dataset. The compounds chosen to
be withheld were the following: i) all small molecules in DrugBank (version
5.0.11) with status ``withdrawn'', for which we could find either a pubchem ID
or a CHEMBL ID, ii)  a randomly selected subset of the remaining compounds in
DrugBank 5.0.11, with status ``approved'', for which we could also find PubChem or CHEMBL IDs,
until a total number of 1\,000 compounds was reached.  No regard was paid to
other drug statuses in DrugBank statuses such as ``investigational''.

\todo{Add info about comparison of results for fillup vs non-fillup? YES! /Ola}

The models built were validated by predicting the binding activity against each
of the 31 targets for all compounds for which we had known binding
data for a particular target. The validation was done with CPSign's validate
command, predicting values at confidence levels 0.8 and 0.9.

\todo{where are these results?}

\section*{Results and Discussion}

\subsection{Before and after fill up with assumed non-actives}

Metrics for all models before fill up with assumed non-actives for some
targets, is shown in figure~\ref{fig:allmodels_nofillup}. In figure
\ref{fig:allmodels_fillup}, the same metrics are shown, after fillup with
assumed non-actives for the 21 targets with less than 10\,000 non-actives in
ExcapeDB.

\begin{figure}[h!]
\includegraphics[width=\textwidth]{figures/final_models_20171106_nofillup.png}
    \caption{All models without fill up with assumed non-actives. The targets are sorted by total number of compounds in ExcapeDB.}
    \label{fig:allmodels_nofillup}
\end{figure}

\begin{figure}[h!]
\includegraphics[width=\textwidth]{figures/final_models_20180223_fillup.png}
    \caption{All models after fill up with assumed non-actives. The targets are sorted by total number of compounds in ExcapeDB. The purple line in the figure shows accuracy.}
    \label{fig:allmodels_fillup}
\end{figure}

\subsection{Observed fuzziness}

In figure \ref{fig:allmodels}, performance metrics for each model is
presented in a plot. The plot shows Observed Fuzziness (OF), Class-Averaged OF
(CAOF) as described in the methods section, training time and validation.
\inlinetodo{What can be seen from the figures?}

\begin{figure}[h!]
\includegraphics[width=\textwidth]{figures/allmodels_wo_drugbank.pdf}
    \caption{All models with the smaller ones being filled up with assumed non-actives.}
    \inlinetodo{Extend the figure caption!}
    \label{fig:allmodels}
\end{figure}



\subsection{Changes in class memberships}

In figure \ref{fig:validation_plots} the class membership change for the
full withheld dataset used in validation, for confidence levels 0.8 and 0.9
respectively.

\begin{figure}[h!]
\hfill
\begin{minipage}{0.45\textwidth}
    \subbottom[Class membership changes for all targets at confidence level 0.8]{\includegraphics[width=1\textwidth]{figures/validation_plots/alltargets_0p8_valplot.pdf}}
\end{minipage}
\hfill
\begin{minipage}{0.45\textwidth}
    \subbottom[Class membership changes for all targets at confidence level 0.9]{\includegraphics[width=1\textwidth]{figures/validation_plots/alltargets_0p9_valplot.pdf}}
\end{minipage}
\hfill
    \caption{Class membership change for all targets, for the prediction data, at confidence level 0.8 and 0.9.
    The labels on the X-axis indicate predicted labels (None, Active (A),
    Non-active (N) or Both). Dark grey in the stacked bar plot means the
    original label was Active (A), while light grey means the original label
    was Non-Active (N).
    }
    \label{fig:validation_plots}
    \inlinetodo{This figure needs a legend for the colours}
\end{figure}

\subsection*{The profile as a web page}
The different models were published as microservices on an OpenShift cluster
and a web page federating all the models were created. At the top of the web
page (See figure~\ref{fig:web}) is and instance of the JSME editor~\cite{Bienfait2013} in which the user
can draw a molecule. As the user draws the molecule the web page will extract
the SMILES from the editor and send it away to the model services and make
predictions based on the QSAR models for the drawn molecule. The user can set a
threshold for the confidence and get visual feedback on whether the models
predict the drawn molecule as active or non active with the chosen confidence.
To the right is also a graphical profile in the form of a bar plot where
confidence as binding is drawn in the positive direction and confidence for
non-binding is the other direction. Hovering it will give information about
exactly which model a certain peak in the bar charts corresponds to. The web
page is accessible at
\url{http://modelingweb.service.pharmb.io/predict/profile/ptp-wo-drugbank}.
\begin{figure}
\begin{wideMinipage}
    \begin{minipage}{0.8\textwidth}
    \subbottom[The profile as it looks on the web page. The user draws a molecule, selects a confidence and then the profile updates underneath.]{\includegraphics[height=0.41\textheight]{figures/terbutaline.png}}
    \end{minipage}
    \begin{minipage}{0.19\textwidth}
    \subbottom[Coloring of which parts of the molecule contributed the most to the prediction for ADBR2.]{\fbox{\includegraphics[height=0.22\textheight]{figures/terbutaline2.png}}}
    \end{minipage}
    \caption{The prediction profile for Terbutaline, a known selective beta-2 adrenergic agonist  used as a broncho\-dilator and tocolytic \label{fig:web}}
\end{wideMinipage}
\end{figure}

\FloatBarrier
\section*{Conclusion}


\section*{Conflict of Interest Statement}
%All financial, commercial or other relationships that might be perceived by
%the academic community as representing a potential conflict of interest must
%be disclosed. If no such relationship exists, authors will be asked to confirm
%the following statement:
OS, JA, AB, and SA are involved in Genetta Soft AB, a Swedish based company developing the CPSign software.


\section*{Author Contributions}
OS conceived the study. OS, JA, SA and SL designed the study, interpreted results, and wrote the manuscript. SL implemented the workflow and carried out the analysis. SA extended CPSign with new features. JA, SA and AB contributed with model deployment and APIs. EA contributed with expertise in target profiles and cheminformatics. All authors read and approved the manuscript.


%The Author Contributions section is mandatory for all articles, including
%articles by sole authors. If an appropriate statement is not provided on
%submission, a standard one will be inserted during the production process. The
%Author Contributions statement must describe the contributions of individual
%authors referred to by their initials and, in doing so, all authors agree to be
%accountable for the content of the work. Please see
%\href{http://home.frontiersin.org/about/author-guidelines#AuthorandContributors}{here}
%for full authorship criteria.

\section*{Funding}
%Details of all funding sources should be provided, including grant numbers if
%applicable. Please ensure to add all necessary funding information, as after
%publication this is no longer possible.
This study was supported by OpenRiskNet (Grant Agreement 731075), a project funded by the European Commission under the Horizon 2020 Programme.

\section*{Acknowledgments}
The computations were performed on resources provided by SNIC through Uppsala
Multidisciplinary Center for Advanced Computational Science (UPPMAX) under
Project SNIC 2017/7-89.
%This is a short text to acknowledge the contributions of specific colleagues,
%institutions, or agencies that aided the efforts of the authors.

\bibliographystyle{unsrt}
\bibliography{ptp}
\newpage
\appendix
\section*{Supplemental Data}
%\href{http://home.frontiersin.org/about/author-guidelines#SupplementaryMaterial}{Supplementary
%Material} should be uploaded separately on submission, if there are
%Supplementary Figures, please include the caption in the same file as the
%figure. LaTeX Supplementary Material templates can be found in the Frontiers
%LaTeX folder
%
%

Supplemental 1: Calibration plots for all targets. See figure
\ref{fig:calibration_plots_all}.

\begin{figure}[h!]
\includegraphics[width=0.19\textwidth]{figures/calibration_plots/ache_calib.pdf}
\includegraphics[width=0.19\textwidth]{figures/calibration_plots/adora2a_calib.pdf}
\includegraphics[width=0.19\textwidth]{figures/calibration_plots/adrb1_calib.pdf}
\includegraphics[width=0.19\textwidth]{figures/calibration_plots/adrb2_calib.pdf}
\includegraphics[width=0.19\textwidth]{figures/calibration_plots/ar_calib.pdf}
\includegraphics[width=0.19\textwidth]{figures/calibration_plots/avpr1a_calib.pdf}
\includegraphics[width=0.19\textwidth]{figures/calibration_plots/cckar_calib.pdf}
\includegraphics[width=0.19\textwidth]{figures/calibration_plots/chrm1_calib.pdf}
\includegraphics[width=0.19\textwidth]{figures/calibration_plots/chrm2_calib.pdf}
\includegraphics[width=0.19\textwidth]{figures/calibration_plots/chrm3_calib.pdf}
\includegraphics[width=0.19\textwidth]{figures/calibration_plots/cnr1_calib.pdf}
\includegraphics[width=0.19\textwidth]{figures/calibration_plots/cnr2_calib.pdf}
\includegraphics[width=0.19\textwidth]{figures/calibration_plots/drd1_calib.pdf}
\includegraphics[width=0.19\textwidth]{figures/calibration_plots/drd2_calib.pdf}
\includegraphics[width=0.19\textwidth]{figures/calibration_plots/ednra_calib.pdf}
\includegraphics[width=0.19\textwidth]{figures/calibration_plots/htr1a_calib.pdf}
\includegraphics[width=0.19\textwidth]{figures/calibration_plots/htr2a_calib.pdf}
\includegraphics[width=0.19\textwidth]{figures/calibration_plots/kcnh2_calib.pdf}
\includegraphics[width=0.19\textwidth]{figures/calibration_plots/lck_calib.pdf}
\includegraphics[width=0.19\textwidth]{figures/calibration_plots/maoa_calib.pdf}
\includegraphics[width=0.19\textwidth]{figures/calibration_plots/nr3c1_calib.pdf}
\includegraphics[width=0.19\textwidth]{figures/calibration_plots/oprd1_calib.pdf}
\includegraphics[width=0.19\textwidth]{figures/calibration_plots/oprk1_calib.pdf}
\includegraphics[width=0.19\textwidth]{figures/calibration_plots/oprm1_calib.pdf}
\includegraphics[width=0.19\textwidth]{figures/calibration_plots/pde3a_calib.pdf}
\includegraphics[width=0.19\textwidth]{figures/calibration_plots/ptgs1_calib.pdf}
\includegraphics[width=0.19\textwidth]{figures/calibration_plots/ptgs2_calib.pdf}
\includegraphics[width=0.19\textwidth]{figures/calibration_plots/scn5a_calib.pdf}
\includegraphics[width=0.19\textwidth]{figures/calibration_plots/slc6a2_calib.pdf}
\includegraphics[width=0.19\textwidth]{figures/calibration_plots/slc6a3_calib.pdf}
\includegraphics[width=0.19\textwidth]{figures/calibration_plots/slc6a4_calib.pdf}

    \caption{Calibration plots for all targets. The plots show accuracy against
        confidence, for confidence values 0.05 to 0.95 with a step size of 0.05.}
    \label{fig:calibration_plots_all}
\end{figure}

Supplemental 2: Class membership change at 0.8 confidence level, for all
targets. See figure \ref{fig:validation_plots_all_0.8}.

\begin{figure}[h!]
\includegraphics[width=0.19\textwidth]{figures/validation_plots/ache_0p8_valplot.pdf}
\includegraphics[width=0.19\textwidth]{figures/validation_plots/adora2a_0p8_valplot.pdf}
\includegraphics[width=0.19\textwidth]{figures/validation_plots/adrb1_0p8_valplot.pdf}
\includegraphics[width=0.19\textwidth]{figures/validation_plots/adrb2_0p8_valplot.pdf}
\includegraphics[width=0.19\textwidth]{figures/validation_plots/alltargets_0p8_valplot.pdf}
\includegraphics[width=0.19\textwidth]{figures/validation_plots/ar_0p8_valplot.pdf}
\includegraphics[width=0.19\textwidth]{figures/validation_plots/avpr1a_0p8_valplot.pdf}
\includegraphics[width=0.19\textwidth]{figures/validation_plots/cckar_0p8_valplot.pdf}
\includegraphics[width=0.19\textwidth]{figures/validation_plots/chrm1_0p8_valplot.pdf}
\includegraphics[width=0.19\textwidth]{figures/validation_plots/chrm2_0p8_valplot.pdf}
\includegraphics[width=0.19\textwidth]{figures/validation_plots/chrm3_0p8_valplot.pdf}
\includegraphics[width=0.19\textwidth]{figures/validation_plots/cnr1_0p8_valplot.pdf}
\includegraphics[width=0.19\textwidth]{figures/validation_plots/cnr2_0p8_valplot.pdf}
\includegraphics[width=0.19\textwidth]{figures/validation_plots/drd1_0p8_valplot.pdf}
\includegraphics[width=0.19\textwidth]{figures/validation_plots/drd2_0p8_valplot.pdf}
\includegraphics[width=0.19\textwidth]{figures/validation_plots/ednra_0p8_valplot.pdf}
\includegraphics[width=0.19\textwidth]{figures/validation_plots/htr1a_0p8_valplot.pdf}
\includegraphics[width=0.19\textwidth]{figures/validation_plots/htr2a_0p8_valplot.pdf}
\includegraphics[width=0.19\textwidth]{figures/validation_plots/kcnh2_0p8_valplot.pdf}
\includegraphics[width=0.19\textwidth]{figures/validation_plots/lck_0p8_valplot.pdf}
\includegraphics[width=0.19\textwidth]{figures/validation_plots/maoa_0p8_valplot.pdf}
\includegraphics[width=0.19\textwidth]{figures/validation_plots/nr3c1_0p8_valplot.pdf}
\includegraphics[width=0.19\textwidth]{figures/validation_plots/oprd1_0p8_valplot.pdf}
\includegraphics[width=0.19\textwidth]{figures/validation_plots/oprk1_0p8_valplot.pdf}
\includegraphics[width=0.19\textwidth]{figures/validation_plots/oprm1_0p8_valplot.pdf}
\includegraphics[width=0.19\textwidth]{figures/validation_plots/pde3a_0p8_valplot.pdf}
\includegraphics[width=0.19\textwidth]{figures/validation_plots/ptgs1_0p8_valplot.pdf}
\includegraphics[width=0.19\textwidth]{figures/validation_plots/ptgs2_0p8_valplot.pdf}
\includegraphics[width=0.19\textwidth]{figures/validation_plots/scn5a_0p8_valplot.pdf}
\includegraphics[width=0.19\textwidth]{figures/validation_plots/slc6a2_0p8_valplot.pdf}
\includegraphics[width=0.19\textwidth]{figures/validation_plots/slc6a3_0p8_valplot.pdf}
\includegraphics[width=0.19\textwidth]{figures/validation_plots/slc6a4_0p8_valplot.pdf}
    \caption{Class membership change at confidence level 0.8, for all targets,
    for the prediction dataset.
    The labels on the X-axis indicate predicted labels (None, Active (A),
    Non-active (N) or Both). Dark grey in the stacked bar plot means the
    original label was Active (A), while light grey means the original label
    was Non-Active (N).
    }
    \label{fig:validation_plots_all_0.8}
\end{figure}

Supplemental 3: Class membership change at 0.9 confidence level, for all
targets. See figure \ref{fig:validation_plots_all_0.9}.

\begin{figure}[h!]
\includegraphics[width=0.19\textwidth]{figures/validation_plots/ache_0p9_valplot.pdf}
\includegraphics[width=0.19\textwidth]{figures/validation_plots/adora2a_0p9_valplot.pdf}
\includegraphics[width=0.19\textwidth]{figures/validation_plots/adrb1_0p9_valplot.pdf}
\includegraphics[width=0.19\textwidth]{figures/validation_plots/adrb2_0p9_valplot.pdf}
\includegraphics[width=0.19\textwidth]{figures/validation_plots/alltargets_0p9_valplot.pdf}
\includegraphics[width=0.19\textwidth]{figures/validation_plots/ar_0p9_valplot.pdf}
\includegraphics[width=0.19\textwidth]{figures/validation_plots/avpr1a_0p9_valplot.pdf}
\includegraphics[width=0.19\textwidth]{figures/validation_plots/cckar_0p9_valplot.pdf}
\includegraphics[width=0.19\textwidth]{figures/validation_plots/chrm1_0p9_valplot.pdf}
\includegraphics[width=0.19\textwidth]{figures/validation_plots/chrm2_0p9_valplot.pdf}
\includegraphics[width=0.19\textwidth]{figures/validation_plots/chrm3_0p9_valplot.pdf}
\includegraphics[width=0.19\textwidth]{figures/validation_plots/cnr1_0p9_valplot.pdf}
\includegraphics[width=0.19\textwidth]{figures/validation_plots/cnr2_0p9_valplot.pdf}
\includegraphics[width=0.19\textwidth]{figures/validation_plots/drd1_0p9_valplot.pdf}
\includegraphics[width=0.19\textwidth]{figures/validation_plots/drd2_0p9_valplot.pdf}
\includegraphics[width=0.19\textwidth]{figures/validation_plots/ednra_0p9_valplot.pdf}
\includegraphics[width=0.19\textwidth]{figures/validation_plots/htr1a_0p9_valplot.pdf}
\includegraphics[width=0.19\textwidth]{figures/validation_plots/htr2a_0p9_valplot.pdf}
\includegraphics[width=0.19\textwidth]{figures/validation_plots/kcnh2_0p9_valplot.pdf}
\includegraphics[width=0.19\textwidth]{figures/validation_plots/lck_0p9_valplot.pdf}
\includegraphics[width=0.19\textwidth]{figures/validation_plots/maoa_0p9_valplot.pdf}
\includegraphics[width=0.19\textwidth]{figures/validation_plots/nr3c1_0p9_valplot.pdf}
\includegraphics[width=0.19\textwidth]{figures/validation_plots/oprd1_0p9_valplot.pdf}
\includegraphics[width=0.19\textwidth]{figures/validation_plots/oprk1_0p9_valplot.pdf}
\includegraphics[width=0.19\textwidth]{figures/validation_plots/oprm1_0p9_valplot.pdf}
\includegraphics[width=0.19\textwidth]{figures/validation_plots/pde3a_0p9_valplot.pdf}
\includegraphics[width=0.19\textwidth]{figures/validation_plots/ptgs1_0p9_valplot.pdf}
\includegraphics[width=0.19\textwidth]{figures/validation_plots/ptgs2_0p9_valplot.pdf}
\includegraphics[width=0.19\textwidth]{figures/validation_plots/scn5a_0p9_valplot.pdf}
\includegraphics[width=0.19\textwidth]{figures/validation_plots/slc6a2_0p9_valplot.pdf}
\includegraphics[width=0.19\textwidth]{figures/validation_plots/slc6a3_0p9_valplot.pdf}
\includegraphics[width=0.19\textwidth]{figures/validation_plots/slc6a4_0p9_valplot.pdf}
    \caption{Class membership change at confidence level 0.9, for all targets,
    for the prediction dataset.
    The labels on the X-axis indicate predicted labels (None, Active (A),
    Non-active (N) or Both). Dark grey in the stacked bar plot means the
    original label was Active (A), while light grey means the original label
    was Non-Active (N).
    }
    \label{fig:validation_plots_all_0.9}
\end{figure}

Supplemental 4: Detailed workflow graph for the fillup-vs-not experiment. See figure \ref{fig:workflow_detailed_fillup_vs_not}.

\begin{figure}[h!]
\includegraphics[width=\textwidth]{figures/workflow_graph_fillup_vs_not.pdf}
    \caption{Detailed workflow graph for the fillup-vs-not experiment.}
    \label{fig:workflow_detailed_fillup_vs_not}
\end{figure}

\end{document}
